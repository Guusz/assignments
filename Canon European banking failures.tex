

\documentclass[DIV=calc, paper=a4, fontsize=11pt, twocolumn]{scrartcl}	 % A4 paper and 11pt font size

\usepackage{lipsum} % Used for inserting dummy 'Lorem ipsum' text into the template
\usepackage[english]{babel} % English language/hyphenation
\usepackage[protrusion=true,expansion=true]{microtype} % Better typography
\usepackage{amsmath,amsfonts,amsthm} % Math packages
\usepackage[svgnames]{xcolor} % Enabling colors by their 'svgnames'
\usepackage[hang, small,labelfont=bf,up,textfont=it,up]{caption} % Custom captions under/above floats in tables or figures
\usepackage{booktabs} % Horizontal rules in tables
\usepackage{fix-cm}	 % Custom font sizes - used for the initial letter in the document

\usepackage{sectsty} % Enables custom section titles
\allsectionsfont{\usefont{OT1}{phv}{b}{n}} % Change the font of all section commands

\usepackage{fancyhdr} % Needed to define custom headers/footers
\pagestyle{fancy} % Enables the custom headers/footers
\usepackage{lastpage} % Used to determine the number of pages in the document (for "Page X of Total")


% Headers - all currently empty
\lhead{Group 10}
\chead{Applied Economic Analysis}
\rhead{European Banking Canon}

% Footers
\lfoot{}
\cfoot{}
\rfoot{\footnotesize Page \thepage\ of \pageref{LastPage}} % "Page 1 of 2"

\renewcommand{\headrulewidth}{0.0pt} % No header rule
\renewcommand{\footrulewidth}{0.4pt} % Thin footer rule

\usepackage{lettrine} % Package to accentuate the first letter of the text
\newcommand{\initial}[1]{ % Defines the command and style for the first letter
\lettrine[lines=3,lhang=0.3,nindent=0em]{
\color{DarkGoldenrod}
{\textsf{#1}}}{}}

%----------------------------------------------------------------------------------------
%	TITLE SECTION
%----------------------------------------------------------------------------------------

\usepackage{titling} % Allows custom title configuration

\newcommand{\HorRule}{\color{DarkGoldenrod} \rule{\linewidth}{1pt}} % Defines the gold horizontal rule around the title

\pretitle{\vspace{-30pt} \begin{flushleft} \HorRule \fontsize{50}{50} \usefont{OT1}{phv}{b}{n} \color{purple} \selectfont} % Horizontal rule before the title

\title{European banking failures} % Your article title

\date{26-10-2016} % Add a date here if you would like one to appear underneath the title block

\posttitle{\par\end{flushleft}\vskip 0.5em} % Whitespace under the title

\preauthor{\begin{flushleft}\large \lineskip 0.5em \usefont{OT1}{phv}{b}{sl} \color{black}} % Author font configuration

\postauthor{\footnotesize \usefont{OT1}{phv}{m}{sl} \color{Black} % Configuration for the institution name
       Tilburg University % Your institution

\par\end{flushleft}\HorRule} % Horizontal rule after the title
%----------------------------------------------------------------------------------------

\begin{document}

\maketitle % Print the title

\thispagestyle{fancy} % Enabling the custom headers/footers for the first page 

%----------------------------------------------------------------------------------------
%	ABSTRACT
%----------------------------------------------------------------------------------------

% The first character should be within \initial{}
\textbf{\section*{Canon Abstract} The aim of this canon is to explain all the main aspects of European banking failures. The function of banks within an economy will be explained as well as the possible causes and consequences of bank failures. Thereby we will provide a brief history and background of banking failures. Furthermore current banking regulations are described and the possible future threats. }

%----------------------------------------------------------------------------------------
%	ARTICLE CONTENTS
%----------------------------------------------------------------------------------------


\begin{table}[]
\centering
\label{my-label}
\begin{tabular}{|l|l|}
\hline
\textbf{Names} & \textbf{ANR}  \\ \hline 
Frank van Casteren & 540947  \\ \hline 
Guus van de Wakker & 722751 \\ \hline
Dian van Rooi & 502916 \\ \hline
Ruben Uijting & 216301 \\ \hline
Job Teurlinx & 943070 \\ \hline
Freek Heuvelmans & 702516 \\ \hline
David Chapa & 693668  \\ \hline
Robbert van Riel & 571548  \\ \hline
\end{tabular}
\end{table}

\section*{Function of banks}
Banks play a key role in our current financial system which in itself is one of the most important factors of economic development. This was expressed as follows by former British prime minister William Gladstone in 1958: "Finance is, as it were, the stomach of the country, from which all the other organs take their tone." (Duisenberg, 2001)

The main function of a bank (Berger, 2014) is to \textit{facilitate a funds flow} from those who have surplus funds to those who have a shortage of funds. This facilitation is needed as the creditors often prefer their assets to be highly liquid, while debtors demand fixed, long-term loans that are illiquid. This difference between liquidity of creditors and debtors is called the a \textit{maturity mismatch} and is one of the main risk factors in bank-based financing. Another advantage of indirect bank-based financing over direct debt financing is the efficiency of the funds allocation. There are often significant \textit{information asymmetries} between borrowers and potential lenders. Banks can overcome information asymmetries to a certain extent due to their comparative advantage in the assessment and monitoring of investment projects. A bank will allocate funds to investment projects with the highest returns, while at the same time minimizing risk, and can do so more efficiently than individual depositors (FRBSF, 2001). This reallocation of capital is most efficient in countries with a highly developed financial system (Duisenberg, 2001). 

The price of this capital is the \textit{interest rate} which is as any other good determined by the supply (depositors) and demand (loans). The \textit{interest rate spread} is the difference between the deposit rate and the lending rate. This spread is generally smaller in countries with a highly developed financial system which allows for more investment projects to be financed and thus more economic growth. A central bank can influence the interest rate through its ability to set the \textit{discount rate}, which is the rate at which banks can borrow from the central bank. These loans are extended for a very short term (usually overnight), to meet short term liquidity needs for depositary institutions in generally sound financial condition (FED, 2016). By its ability to shift the interest rate, Central banks can influence the availability of funds and change the rates of consumer spending and borrowing. By using these monetary policies, central bank can influence short term economic activity.

In order for this whole system of bank-based financing to work it relies on \textit{trust of depositors}. The loss of trust in a bank can lead to bank runs which, as a self-fulfilling prophecy and due to the maturity mismatch, can cause bankruptcy. The effects of such a bankruptcy on the entire financial system depend on the size of the institution and the order of interconnectedness between the different institutions. A government can \textit{bail out} the larger banks to prevent this knock-on effect and the collapse of the financial system. This in turn will create a \textit{moral hazard} for larger banks and financial institutions to take more risk. For this reason, central banks will set \textit{ banking regulations} that should decrease the probability of default and its effects, while still ensuring an efficient reallocation of capital. 

\section*{Causes of banking failures}
Now we know a bit more about the functions of a bank and how it works, it is easier to explain what the causes of a banking failure are. Of course, there is not a single cause of a failure, often a combination of different factors play a role in explaining why a banking failure happened.

\textbf{Maturity Mismatch:}

As seen in the previous sector, banks receive surplus money from people who do not need it at the moment: \textit{deposits} and loan this money to people who do need it \textit{loans}. This function of the bank is also where problems tend to exist (Greenham, 2012). The maturity of a bank's liabilities are often shorter than the maturity of its assets. Deposits are often short term and can be withdrawn by customers at any moment. Loans on the other hand are long term, think about mortgages for example. All kinds of mismatching problems then start to develop. If either the assets drop in value or deposits are withdrawn at a massive rate, the bank will fail.

\textbf{What causes assets to drop in value and deposits to be withdrawn?}

We will start with the first cause. To illustrate this we use the \textit{2008 financial crisis} as an example. Before the crisis interest rates were relatively low, while home prices were increasing. Many people therefore took out a mortgage. This seemed to be a good investment. Even people who were not really able to afford those mortgages were able to get one. As long as home prices increase, there is not really a problem. Then you make a profit and you are sure to finance your mortgage. However, if home prices do not longer increase that is when problems start to arise. Interest rates also started to increase and therefore the monthly payments increased (Greenham, 2012). Especially people who could not afford the mortgage in the first place, but who were not checked by the bank, could not afford the monthly payments. The borrowers therefore partly or even completely default on their loans. The bank's assets are therefore decreased in value and unless the bank has enough cash reserves, the bank is not able to pay off its depositors.

A drop in the value of a bank's assets is not the only way a bank can become \textit{insolvent}. As stated earlier, a withdrawal of liabilities i.e. deposits, can also cause a bank to become insolvent. If many depositors withdraw their money at the same time, this can also be called a bank run. Depositors do this because they believe that the bank is not able to repay their depositors. We showed this in the mortgage example. Deposits as stated before are liquid and can be withdrawn at any moment by the customer. The bank's assets are on the other hand illiquid and long term. A bank usually has some cash reserves but these reserves are dried up quite quickly. If then a lot of deposits are withdrawn at the same time, the bank has a problem. The bank has to sell these illiquid assets, for example mortgages. This is however not possible or it has to sell them at only a fraction of their true value. Again, the bank's assets are then decreased in value and this will drain the capital reserves of the bank. The end result is that the bank will become insolvent.

We have covered two situations which can make a bank fail. One in which the assets drop in value and one in which deposits are withdrawn. There are however many more causes for a banking failure. Bad management and excessive risk-taking to name a few. It is important to understand that a banking failure generally arises from a set of different causes. A lot can be done to prevent banking failures, in our examples, if the bank had more cash reserves the problem would not have been as big. This will however be discussed later.

\section*{Consequences of banking failures}
Banks and financial institutions are important institutions that allocate capital, both from persons who have a supply of capital – usually households or firms -  to firms or households in need of capital. Banks and other financial institutions are also needed in this process, because they have the ability to transform short-term credit to long-term debt for, for instance, mortgages and commercial bonds (Wei Xiong, 2010). If banks fail, however, this process is distorted. The most important consequences are presented here: 

\textbf{Bank runs:}

Banks are unable to fulfill their debt obligations towards firms or households with deposits at the bank. This could, in turn, lead to a bank run, when people with deposits at the bank try to redeem their money, fearing that their deposits will be used to pay off other debitors of the bank. While this behaviour is very understandable, it only worsens the problems of the bank and to prevent this from happening, the government usually intervenes and guarantees they payback of your deposits. If a bank is unable to do so, the government ill do this themselves. Such schemes have been installed in all countries of the EU. The government guarantees only deposits to a certain amount (100 000 in the Netherlands). But, even now deposits of non-commercial parties are protected, the government is responsible for paying these guarantees and also bears the costs of it. 

\textbf{Credit rollovers:}

Firms usually have long-term debt, but occasionally also use short-term debt to finance their capital needs. If banks fail and are no longer able to provide debt to their customers, this will be problematic for firms as well. The failing of a bank can for a firm mean that they have to find other forms of credit with more costs attached. 

\textbf{Systematic risk:}

Banks usually are interdependent and borrow from and lend to other banks and financial institutions (Ellis, 2014). If one bank fails, other banks can also get into trouble to meet their obligations when one or more of their debtors default. The entire financial system can also be threatened if the failing bank is so large that other financial institutions are not able to replace the role of the failing bank instantly. 

\textbf{Credit provision:}

If banks fail and the entire financial system is distorted as a result of this, or no other bank is able to fulfil the functions of the failing bank, then the real economy can suffer from the failing bank, as companies or households can no longer find credit to finance their projects or fulfil their desires. New, innovative companies can no longer find credit to use their ideas to make the economy more efficient and households cannot find money to finance houses, even though they could have the money to pay off their debts later. Economists call this situation sub-optimal, as these actors are not able to do something they would like to do. 

\section*{History of banking failures}
Ever since banks were created the possibility that they could fail has been present. The causes for failure in the first banks were due to mismanagement or external factors that banks did not foresee and were not prepared to face. Hereby a summary of the most important banking failures of the European Banking History based on articles of Laeven in 2008 and of Reinhart in 2009. 

\textbf{Amsterdam Banking Crisis 1763}

One of the first documented cases of a banking failure in Europe is the Amsterdam 
banking crisis of 1763. This crisis came about because the banking sector in the Netherlands had borrowed 
heavily using basic goods as collateral (a guarantee of payment), these goods were sold at a higher price 
than usual due to the Seven Year’s War. When the war was over the prices of the basic goods fell 
dramatically and the amount of money available decreased to a point where intervention by the Bank of 
Amsterdam was required to provide enough money for the banking system. Throughout the rest of the 
18th and 19th centuries most of the banking failures were triggered by mismanagement in banks and/or 
critical loss of confidence from the depositors in the ability of the banks to repay the money that had been 
deposited in their institutions. Notable examples of these situations are the Panics of 1825 and 1866. 

\textbf{London Banking Crisis 1825 and 1866}

In the year of 1825 a stock market crash in London that was caused from reckless investments in Latin 
America caused the collapse of six banks in London, this event is considered to mark the beginning of the 
modern economic cycles due to the fact that the panic was not caused by an external event but rather by 
speculative behavior. In the 1866 crisis, the failure of a bank called Overend, Gurney and Co. again because 
of mismanagement, prompted the failure of over 200 companies. 

\textbf{Great Depression 1929}

In the 20th century the worst banking 
crisis was the Great Depression in United States. This event in 1929 was caused by a crash in the American 
Stock Market. Four years later, 11,000 of United States' 25,000 banks had failed and at the height of the 
Great Depression unemployment had risen to 25%. 

\textbf{Banking Crisis 2008}

In the 21st Century we have already had a severe banking crisis. This crisis was again brought about by mismanagement and lack of oversight from regulating authorities which led to essentially banks betting that housing prices would continue to rise, 
leading to a “housing bubble”. Banks lent money to people who were not credit worthy, and they 
eventually could not pay back, which led to the banks losing money on the money they could not recover 
and “losing their bets” on the housing market. This Sub-Prime Mortgage Crisis, as it became known, 
affected banks all over the world causing severe banking failures in United Kingdom, Belgium, Iceland, 
Spain and Ireland to name a few.

\section*{European banking regulations}
Regulation policy includes three lines of defence that all contribute to the main goal, financial stability. The first line of defence is policy that prevents developments leading to threats of financial stability. An example of first line policy is mandatory extensive due diligence on mortgages. The second line of defence is policy that increases the resilience of the financial system. An example of second line policy is additional capital requirements for financial institutions. The last line of defence is policy decreasing systemic risk, damage limitation. Whenever an institution fails, the damage to the financial system should be as low as possible such that the taxpayer is paying the damage. An example of third line policy is extensive and sound resolution schemes.

After the global financial crisis of 2008 a major consensus was reached concluding that the European banking regulation was not adequate. Some additional regulations were needed. The Basel Committee of Banking Supervision (BCBS) expanded their guidelines of banking supervision. The European Union (EU) implemented these guidelines in EU legislation. 

Important key principles of European banking regulation from the most recent Basel accord (Basel III) are discussed below.  
First of all, capital requirements, banks need to have a minimum common equity ratio that equals 4.5\%. 
This ratio is calculated by total common equity divided by the total amount of risk-weighted assets. 

 \begin{equation} \label{eq1}
\begin{split}
4.5\% & \geq \frac{Common Equity}{Risk Weighted Assets}
\end{split}
\end{equation}

Moreover a capital ratio of 6\% is demanded. This capital is the sum of common equity, preferred stocks and non-controlling interests. As an addition, a counter cyclical common equity buffer is required; this additional buffer can be up to 2.5\%. In Europe, the implementation of this counter cyclical buffer is not fully implemented yet.

Another key principle is the leverage ratio. This implies that the amount of common equity must be 3\% of total risk exposure. Total risk exposure is exposure of all balance-sheet assets and off-balance sheet risks. In Europe, the leverage ratios of banks are supervised, but not mandatory yet.

The last key principle of banking regulation in the EU is a set of liquidity requirements (Basel Committee, 2010). A bank is pressurized to have a certain amount of high-quality liquid assets. This amount of liquid assets should equal the net cash outflows over 30 days. The second liquidity requirement is a Net Stable Funding Ratio (NSFR). This ratio, the available amount of stable funding must be higher than the minimum required amount of stable funding over a one-year period of extended stress.

\section*{Future Threats}
Banks are subject to a lot of factors that influence the bank’s performance. This section explains what the most important threats to the current banking system are. 

\textbf{Current European problems}

The European debt crisis started end 2009 with the inability of some countries in Europe to finance their high debt-levels. This meant that large banks couldn’t be bailed out anymore in times of trouble, and that the amount of loans payed back to banks dropped. Although financial stability is restoring in these countries, it is still uncertain what happens to the financial system if an important bank has to be bailed out. The main message is that if banks fail in Europe, this will cause a shock of uncertainty about all banks in Europe.

\textbf{Deutsche Bank}

Deutsche Bank is the biggest bank of Germany and has a yearly revenue of approximately 47.4 billion US dollars and 1.7 trillion in assets. The bank came into trouble in July 2015 when the US government announced that the bank will be fined for her role in the sub-prime mortgage crisis in 2008 (Inman, 2016). These fines put pressure on the financial wealth of the bank, therefore also the ability to pay for outstanding debt. Due to the high amount of fines, investors have pulled out money from Deutsche Bank, so that it lost more than half of its market value in one year. Since banks have interests in other banks, and they lend out money overnight to each other, this creates a certain interconnectedness. The figure below illustrates the interconnectedness of Deutsche Bank with other big banks around the world. Obviously, if Deutsche Bank falls, this will have a huge impact on the performance of many other banks which are both directly and indirectly linked. 

It’s not only Deutsche Bank that faces problems. Other European banks, mainly Italian, are struggling with outstanding loans that do not get paid back to the bank. This, too, puts financial pressure on the banks. Given the interconnectedness of banks and the problems of some EU governments, this is a big threat.

\textbf{Divided EU}

With Brexit happening, what happens with the banking sector if more countries would leave the European Union. Recently, some multi-nationals have rumored that they might leave London due to the unfavorable expected business climate in a non-EU Britain. This business environment is unfavorable since markets are less open and the cost of capital transportation is relatively higher. This raises the question what happens if more countries would leave the EU. Italy, for example, is having a referendum in December whether to reform the political structure in the country. The fear is that if the referendum ends up being a “no”, the current Prime Minister Renzi will resign. Since the Five-star party in Italy is very popular at the moment, they will be likely to be elected in the new government. One of the main points of this party’s agenda is to leave Europe. This will create only more uncertainty about the future of Europe, and the attractiveness to be active in Europe as a financial institution. 

\textbf{Interest rate}

Another problem that European banks face nowadays is the low interest rate. Banks are less able to make profits since the investment return has lowered since the European Central Bank (ECB) has lowered the interest rate in 2012 (Constancio, 2016). The ECB has lowered the interest rate to increase inflation to the desired level of 2%. Since this it is not likely that this happens any time soon, European banks have to deal with these low rates, which is a challenge and puts more pressure on banks.

\textbf{Financial service development}

With the start of the consumer digitalization in the 1990s, the opportunities arose to optimize processes and communication has increased at an incredible rate. Nowadays, there are many financial technology companies (FinTechs) that substitute services of the banks. According to McKinsey, in the services that banks provide, 20%-60% of the profits and 10%-40% of the revenue will at risk by 2025 due to these FinTechs. For example, instead of expensive fund managers and investment funds, FinTech companies provide comparable services for a fraction of the costs that banks charge. These firms develop programs and use big data to invest consumer money in a similar way that banks would do. 

Another factor that is hugely affected by the digitalization is the provision of funds from-and-to consumers by companies. These parties no longer need the bank for funds, but can directly find a counterparty to exchange funds with. This process eliminates the need for banks, which is, obviously, a risk for banks. Social media enables fast information exchange, which might both negatively or positively affect bank reputation. Banks now have less time to react to problems due to the increase of communication speed. Obviously, banks still hold a strong position when it comes to expertise in the financial sector and information resources. Although we see that banks hold a strong position in the provision of liquidity and financial services in the economy, the competition in financial services increases rapidly. The future profit and revenue of banks might be at risk due to these FinTechs and digital innovations.

\textbf{Cyber criminality}

In line with the opportunities that the digitalization has to offer, this also has a downside. Computer criminals now have an extra dimension to rob money from institutions of individuals that have much funds. Banks encounter cyber criminality daily. In 2015, a group of hackers called ‘Businessclub’ stole approximately 100 million dollar from bank accounts by hacking the servers of banks. The cost of a cyber-attack is not only the money that is lost to hackers, but also retrieving information concerning the breach is costly, the impact on the bank’s brand reputation is costly since consumers lose faith due to social media, and there might be regulatory fines that lowers revenue and profit. Cyber security is getting more and more important.

%----------------------------------------------------------------------------------------
%	REFERENCE LIST
%----------------------------------------------------------------------------------------

\begin{thebibliography}{99} % Bibliography - this is intentionally simple in this template
\bibitem [1]{}
\newblock {Berger, A. N., Molyneux, P., and Wilson, J. O. (Eds.). (2014). The Oxford handbook of banking. OUP Oxford.}
\bibitem [2]{}
\newblock {Duisenberg, W. F. (2001). The role of financial markets for economic growth. Vienna: Economics Conference "The Single Financial Market: Two Years into EMU".}
\bibitem [3]{}
\newblock {Federal Reserve Bank Of San Fransisco (2001, July). What is the economic function of a bank. Retrieved October 25, 2016, from frbsf.org.}
\bibitem [4]{}
\newblock {Federal Reserve System (2016). The Discount Rate. Retrieved October 25, 2016, from www.federalreserve.gov.}
\bibitem [5]{}
\newblock {Greenham, T., Ryan-Collins, J., Werner, R., and Jackson, A. (2012). Where does money come from?: a guide to the UK monetary and banking system. The New Economics Foundation.}
\bibitem [6]{}
\newblock {Wei Xiong and Zhiguo He, (2010). Rollover Risk and Credit Risk, 2010 Meeting Papers 98, Society for Economic Dynamics. }
\bibitem [7]{}
\newblock { Ellis, Luci and Haldane, Andy and Moshirian, Fariborz, 2014. "Systemic risk, governance and global financial stability," Journal of Banking and Finance, Elsevier, vol. 45(C), pages 175-181}
\bibitem [8]{}
\newblock {Laeven L, Valencia F (2008). "Systemic banking crises: a new database". IMF WP/08/224. International Monetary Fund}
\bibitem [9]{}
\newblock {This Time It’s Different: Eight Centuries of Financial Folly, Carmen Reinhart and Kenneth Rogof 2009 University of Maryland, College Park, Department of Economics, Harvard University}
\bibitem [10]{}
\newblock {Basel Committee. (2010). Basel III: A global regulatory framework for more resilient banks and banking systems.}
\bibitem [11]{}
\newblock { Basel Committee on Banking Supervision, Basel. OJ L 176, 27.6.2013, p. 338–436 Special edition in Croatian: Chapter 06 Volume 014 P. 105 - 203}
\bibitem [12]{}
\newblock {Inman, P., and Davies, R. (2016, February 10). The five fears stalking the global banking industry. Retrieved October 25, 2016, from https://www.theguardian.com/business/2016/ feb/10/banking-shares-under-pressure-as-investors-fear-effects-of-global-downturn}
\bibitem [13]{}
\newblock {Constancio, V. (2016, July 7). Challenges for the European banking industry. Retrieved October 25 2015, from https://www.ecb.europa.eu/press/key/date/2016}
\end{thebibliography}

%----------------------------------------------------------------------------------------

\end{document}